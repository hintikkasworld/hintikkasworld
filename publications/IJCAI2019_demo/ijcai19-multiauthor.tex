%%%% ijcai19-multiauthor.tex

\typeout{IJCAI-19 Multiple authors example}

% These are the instructions for authors for IJCAI-19.

\documentclass{article}
\pdfpagewidth=8.5in
\pdfpageheight=11in
% The file ijcai19.sty is NOT the same than previous years'
\usepackage{ijcai19}

% Use the postscript times font!
\usepackage{times}
\usepackage{soul}
\usepackage{url}
\usepackage[hidelinks]{hyperref}
\usepackage[utf8]{inputenc}
\usepackage[small]{caption}
\usepackage{graphicx}
\usepackage{amsmath}
\usepackage{booktabs}
\usepackage{tikz}
\usepackage{tikz-uml}

\urlstyle{same}

% the following package is optional:
%\usepackage{latexsym} 

% Following comment is from ijcai97-submit.tex:
% The preparation of these files was supported by Schlumberger Palo Alto
% Research, AT\&T Bell Laboratories, and Morgan Kaufmann Publishers.
% Shirley Jowell, of Morgan Kaufmann Publishers, and Peter F.
% Patel-Schneider, of AT\&T Bell Laboratories collaborated on their
% preparation.

% These instructions can be modified and used in other conferences as long
% as credit to the authors and supporting agencies is retained, this notice
% is not changed, and further modification or reuse is not restricted.
% Neither Shirley Jowell nor Peter F. Patel-Schneider can be listed as
% contacts for providing assistance without their prior permission.

% To use for other conferences, change references to files and the
% conference appropriate and use other authors, contacts, publishers, and
% organizations.
% Also change the deadline and address for returning papers and the length and
% page charge instructions.
% Put where the files are available in the appropriate places.

\title{Hintikka's World: scalable higher-order knowledge}

\author{
Tristan Charrier$^1$\and
Sébastien Gamblin$^2$\and
Alexandre Niveau$^{2,3}$\And
François Schwarzentruber$^4$\footnote{Contact Author}\\
\affiliations
$^1$Univ Rennes, CNRS, IRISA, France\\
$^2$Université de Caen Normandie, GREYC, Caen, France\\
$^3$Université de Caen Normandie, GREYC, Caen, France\\
$^4$Univ Rennes, CNRS, IRISA, France\\
\emails
tristan.charrier@irisa.fr, 
sebastien.gamblin@unicaen.fr
alexandre.niveau@unicaen.fr,
francois.schwarzentruber@ens-rennes.fr
}

\begin{document}
\newcommand{\mettel}{\textsf{MetTeL2}\xspace}

\maketitle

\begin{abstract}
	\emph{Hintikka's World} is a tool that shows how artificial agents can reason about higher-order knowledge (agent $a$ knows that agent $b$ knows that...).
	In this demonstration paper, we present symbolic models  that enables to implement in  \emph{Hintikka's World} large examples such as real card games. 
\end{abstract}



\section{Introduction}
The current trend is to construct programs that play games with imperfect information, for instance Hanabi \cite{DBLP:journals/corr/abs-1902-00506}, but also video games such as Starcraft 2 \cite{DBLP:conf/ijcai/HuLLPX18} An important ingredient is to reason about higher-order knowledge (an agent knows that another agents knows that...). That is why we claim that epistemic logic and its dynamic extension, Dynamic epistemic logic (\cite{baltag1998logic}, \cite{DitmarschvdHoekKooi}), may offer a formal tool to provide explanation in such AI programs. needs to be understood is relevant in AI, especially in strategic reasoning \cite{DBLP:journals/ijgt/Aumann99}.

The only tool we are aware of that enables to see and explore mental states of agents is \emph{Hintikka's world} and was presented at ECAI-IJCAI 2018 \cite{DBLP:conf/ijcai/Schwarzentruber18}. 
\emph{Hintikka's world} is a proof of concept of a graphical user interface that represent Kripke models by  comic strips, as shown in Figure \ref{figure:gui}. The tool is available at the following address:
\url{http://hintikkasworld.irisa.fr/}. 


Kripke models are graphs and they were represented explicitly in memory in the first version of the tool. Explicit models are useful to learn how dynamic epistemic logic works by means of toy examples: muddy children, Sally and Anne  \cite{wimmer1983beliefs}, etc.  However, in real card games, such as Hanabi, there are .... possible configurations of cards. Thus, it is impossible to represent the full graph in memory:  the first version of Hintikka's world does not \emph{scale}. 

That is why, we proposed to represent Kripke models symbolically by means of Binary Decision Diagrams as it was done in the tool DEMO\footnote{The current implementation does not rely on DEMO since their work is not well-suited for a web application.}  \cite{DBLP:conf/lori/BenthemEGS15}. 





\begin{figure}
	\begin{center}
		\includegraphics[width=4cm]{screenshot.png}
	\end{center}
	\caption{Graphical user interface of \emph{Hintikka's world}\label{figure:gui}}
\end{figure}


\section{Demonstration Outline}
\label{section:demonstration}

In the demonstration we run through a game of (a variant of) Hanabi. 
In Hanabi, each agent has cards with a color and a
number, but cannot see his/her own hand.
At each turn, in \emph{Hintikka's World}, the user can play the role of one of the agents: he/she can either give the information to some other agent about a number or a color, or play a card. The goal is to play the cards in increasing order for each color.
During the process, the system keeps track on the knowledge of the agents.
More precisely, the system shows the real world (the real distribution of the cards). When the user clicks an agent, the system displays a \emph{sampling} of some possible worlds for that agent (i.e., some distributions of cards he/she still considers as possible at this stage of the game). The agents also reason about knowledge of other agents, as shown in Figure~\ref{figure:guihanabi} (two levels of knowledge are shown).
%
%
%% expliquer la démo. Je me suis peut etre trompé sur le nombre de cartes ici et si on gère les jetons pour les infos il faut le mettre dans le paragraphe.
% Alex: pas grave au pire. on dit qu'on peut faire ça ou ça, c'est pas _forcément_ exclusif.
%


\begin{figure}
	\begin{center}
		\includegraphics[width=4cm]{images/HW_screenshot_hanabi.png}
	\end{center}
\vspace{-3mm}
	\caption{Screenshot of Hanabi in \emph{Hintikka's World}.\label{figure:guihanabi}}
\end{figure}
%%\newpage

Note that in this demonstration, in order to explain models of DEL, the tool still presents examples that rely on explicit models, such as ``Sally and Anne'', ``Muddy Children'', ``Consecutive Numbers'', etc.



\section{Symbolic models}
\label{section:symbolicmodels}
A symbolic model consists in a description of the model in a more succinct language. Such representations already exist for other formalisms, such as boolean circuits for searching for Hamiltonian paths in graphs \cite{papadimitriou2003computational}, or BDDs for temporal logics \cite{DBLP:conf/lics/BurchCMDH90}.

In the demonstration we use the symbolic description of \cite{DBLP:conf/aiml/CharrierS18} for dynamic epistemic logic \footnote{Which is actually a rewriting of the description of \cite{DBLP:conf/atal/CharrierS17}}, and use the reduction to first-order logic of \cite{DBLP:conf/tableaux/CharrierPS17} to implement this approach with BDDs.

Take for instance any card game. The initial model is described by BDDs for the following boolean formulas: formula $\chi_W$  whose valuations are the possible configurations and formula $\chi_R$ to describe which configurations agents do not distinguish. The formula $\chi_W$ actually describes legal configurations according to the rules of the game, and $\chi_W$ reflect the fact that agents do not see the hand of other agents, so they may imagine as possible any possible dealing.







\section{System Description}
\label{section:architecture}

Whereas the first version was written in JavaScript, in order to ease the development, the new version is written in TypeScript and relies on the Angular~7 framework.

\subsection{Binary decision diagrams}

As shown by \citet{DBLP:conf/atal/CharrierS17}, the symbolic model checking of DEL is PSPACE-complete, thus is critical. We manipulate sets of worlds as well as relations by means of Binary Decision Diagrams. To this aim, we wrote a JavaScript wrapper of the C library CUDD (Colorado University Decision Diagram Package)~\cite{DBLP:journals/sttt/Somenzi01}: we wrote a thin wrapper in C, which we compiled into Web Assembly using Emscripten, in order to be usable from our JavaScript module.

In order to show possible worlds for a given agent $a$ in some world $w$, we first construct the BDD of $\succinctrelation a(descr(w), \vec x')$ where $descr(w)$ are the Boolean values of $\vec x$ corresponding to world $w$. We then count the number of possible valuations $\vec x'$ that make $\succinctrelation a(descr(w), \vec x')$ true (BDDs are an efficient representation for counting valuations satisfying a Boolean formula). If the number of such valuations is small, we show all possible worlds, otherwise we randomly generate valuations for $\vec x'$ that makes $\succinctrelation a(descr(w), \vec x')$ true (we randomly select a branch that leads to the ``true'' leaf in the BDD of $\succinctrelation a(descr(w), \vec x')$).

\subsection{Class Architecture}

Figure \ref{figure:architecture} shows the new architecture of \emph{Hintikka's world}. \texttt{EpistemicModel} is an abstract class, used by the graphical user interface (GUI), that is independent from the concrete example (``Muddy Children'', ``Sally and Anne'', ``Hanabi'', etc.) but also, more interestingly, independent from the representation of the epistemic model itself. In particular, an epistemic model can be an \texttt{ExplicitEpistemicModel} (a graph) or a \texttt{SymbolicEpistemicModel} that relies on BDDs. To obtain a comic strips for a given example, it suffices to implement the method \texttt{draw} of a class that inherits from class \texttt{World}, that tells how a possible world is drawn.


\subsection{Adding new examples}

Providing new examples is easy with our system. Explicit epistemic models are simply directly described (sets of nodes and of edges). Symbolic epistemic models are described by a Boolean formula $\succinctsetworlds$, or Boolean formulas for $\succinctrelation{a}$. The system provides a way to easily describe how worlds are displayed in the comic strips.

\begin{figure}
	\begin{center}
		\scalebox{0.6}{
			\begin{tikzpicture}[scale=0.75]
			
			\umlclass[x=0,y=2]{EpistemicModel}{
				
			}{}
			
			\umlclass[x=-6,y=2]{Graph}{
			}{
			}
			
%			\umlclass[x=-2,y=-2.5]{World}{
%			}{
%			}

	\umlclass[x=8,y=2]{BDD}{
}{
}
			
			\umlclass[x=-3,y=-0]{ExplicitEpistemicModel}{
			}{
			}
			
			\umlclass[x=4,y=-2.4]{Hanabi}{
			}{
			}
			
			\umlclass[x=-3,y=-2.4]{SallyAndAnne}{
			}{
			}
			
			\umlclass[x=4,y=-0]{SymbolicEpistemicModel}{
			}{
			}
			
			
		%	\umlassoc[geometry=--, arg1=, mult1=1, align1=right, arg2=, mult2=*, align2=left]{GUI}{EpistemicModel}
		%	\umlassoc[geometry=--, arg1=, mult1=*, arg2=, mult2=1]{EpistemicModel}{World}
			\umlassoc[geometry=--, arg1=, mult1=, arg2=, mult2=1]{ExplicitEpistemicModel}{SallyAndAnne}
			\umlassoc[geometry=--, arg1=, mult1=, arg2=, mult2=1]{SymbolicEpistemicModel}{Hanabi}
			\umlinherit[geometry=|-]{ExplicitEpistemicModel}{Graph}
			\umlinherit[geometry=|-]{SymbolicEpistemicModel}{EpistemicModel}
			\umlinherit[geometry=|-]{ExplicitEpistemicModel}{EpistemicModel}
			\umlassoc[geometry=-|, arg1=, mult1=*, arg2=, mult2=]{SymbolicEpistemicModel}{BDD}			

			
			%\umlunicompo[geometry=-|, arg=titi, mult=*, pos=1.7, stereo=vector]{D}{C}
			%\umlaggreg[arg=tutu, mult=1, pos=0.8, angle1=30, angle2=60, loopsize=2cm]{D}{D}
			
			\end{tikzpicture}}
		\vspace{-3mm}
	\end{center}
	\caption{New architecture of \emph{Hintikka's world}.\label{figure:architecture}}
\end{figure}


\section{Future Work}
\label{section:perspectives}

TODO implémzenter d'autres exemples etc.


TODO d'autres façons de "scaler" (parler des ATD)

TODO parler de méthodes statistiques pour générer des mondes possibles










\newpage



\bibliographystyle{named}
\bibliography{biblio}


\newpage

\section*{List of requirements/description of the demo setting}

\begin{itemize}
	\item Table, poster, monitor.
\end{itemize}

\end{document}

