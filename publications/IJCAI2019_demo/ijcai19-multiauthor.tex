%%%% ijcai19-multiauthor.tex

\typeout{IJCAI-19 Multiple authors example}

% These are the instructions for authors for IJCAI-19.

\documentclass{article}
\pdfpagewidth=8.5in
\pdfpageheight=11in
% The file ijcai19.sty is NOT the same than previous years'
\usepackage{ijcai19}

% Use the postscript times font!
\usepackage{times}
\usepackage{soul}
\usepackage{url}
\usepackage[hidelinks]{hyperref}
\usepackage[utf8]{inputenc}
\usepackage[small]{caption}
\usepackage{graphicx}
\usepackage{amsmath}
\usepackage{booktabs}
\usepackage{tikz}
\usepackage{tikz-uml}

\urlstyle{same}

% the following package is optional:
%\usepackage{latexsym} 

% Following comment is from ijcai97-submit.tex:
% The preparation of these files was supported by Schlumberger Palo Alto
% Research, AT\&T Bell Laboratories, and Morgan Kaufmann Publishers.
% Shirley Jowell, of Morgan Kaufmann Publishers, and Peter F.
% Patel-Schneider, of AT\&T Bell Laboratories collaborated on their
% preparation.

% These instructions can be modified and used in other conferences as long
% as credit to the authors and supporting agencies is retained, this notice
% is not changed, and further modification or reuse is not restricted.
% Neither Shirley Jowell nor Peter F. Patel-Schneider can be listed as
% contacts for providing assistance without their prior permission.

% To use for other conferences, change references to files and the
% conference appropriate and use other authors, contacts, publishers, and
% organizations.
% Also change the deadline and address for returning papers and the length and
% page charge instructions.
% Put where the files are available in the appropriate places.

\title{Hintikka's World: scalable higher-order knowledge}

\author{
Tristan Charrier$^1$\and
Sébastien Gamblin$^2$\and
Alexandre Niveau$^{2}$\And
François Schwarzentruber$^3$\footnote{Contact Author}\\
\affiliations
$^1$Univ Rennes, CNRS, IRISA, France\\
$^2$Université de Caen Normandie, GREYC, Caen, France\\
$^3$Univ Rennes, CNRS, IRISA, France\\
\emails
tristan.charrier@irisa.fr, 
sebastien.gamblin@unicaen.fr,
alexandre.niveau@unicaen.fr,
francois.schwarzentruber@ens-rennes.fr
}

\begin{document}
\newcommand{\mettel}{\textsf{MetTeL2}\xspace}
\newcommand{\citet}[1]{\citeauthor{#1}~\shortcite{#1}}

\maketitle

\begin{abstract}
	Hintikka's World is a graphical and pedagogical tool that shows how artificial agents can reason about higher-order knowledge (agent $a$ knows that agent $b$ knows that...).
	In this demonstration paper, we present the implementation of symbolic models in Hintikka's World. They enable the tool to scale, by helping it to face the state explosion, which makes it possible to provide examples featuring real card games, such as Hanabi.
\end{abstract}



\section{Introduction}
\input{section_introduction.tex}

\section{Demonstration Outline}
\label{section:demonstration}
\input{section_demonstrationoutline}


\section{Symbolic models}
\label{section:symbolicmodels}
\input{section_symbolicmodels}







\section{System Description}
\label{section:architecture}
\input{section_systemdescription}

\section{Future Work}
\label{section:perspectives}
\input{section_conclusion}

\paragraph{Acknowledgments. }
%
We thank Hai Trung Pham for his work during his internship and Arthur Queffelec for his comments on the architecture.








\newpage



\bibliographystyle{named}
\bibliography{biblio}

\end{document}

